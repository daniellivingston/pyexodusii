\chapter{Site Supplements} \label{a:site}


\section{VAX VMS}

The command to execute \caps{\numbers} on the \SNL\ Engineering Sciences VAXCluster
running the \caps{VMS} operating system is: 
\begin{center}
\cmd{numbers}\ \param{database}\ \optparam{user\_input}
\end{center}

\param{database} is the filename of the input \exo\ 
database. A prompt appears if \param{database} is omitted. 

If \param{user\_input} is given, the user input is read from this
file. Otherwise it is read from \caps{SYS\$INPUT} (the terminal
keyboard). User output is directed to \caps{SYS\$OUTPUT} (the terminal).

\caps{\numbers} operates in either interactive or batch modes.

\section{CRAY CTSS}

To execute \caps{\numbers}, the user must have selected the Engineering
Analysis Department's \cmd{acclib} library and be running \cmd{ccl}. 

The command to execute \caps{\numbers} on \caps{CTSS} is:
\begin{center}
\sf numbers\ \param{database}\ i=\param{input}\ o=\param{output}
\end{center}

\param{database} is the filename of the input \exo\ 
database. The default is {\sf tape9}.

User input is read from \param{input}, which defaults to {\sf tty} (the
terminal). User output is directed to \param{output}, which defaults to
{\sf tty} (the terminal). 


\section{Execution Files} \label{exec:files}

The table below summarizes the files used by \numbers.

\begin{center} \begin{tabular}{||l|c|c|c||}
\hline
\multicolumn{1}{||c}{Description} &
\multicolumn{1}{|c}{Unit Number} &
\multicolumn{1}{|c}{Type} &
\multicolumn{1}{|c||}{File Format} \\
\hline
     User input & standard input  & input  & Section~\ref{c:commands} \\
    User output & standard output & output & ASCII \\
Output file     &               7 & output & ASCII \\
EXODUS database &               9 & input  & \\
\hline
\end{tabular} \end{center}

All files must be connected to the appropriate unit before
\numbers\ is run. Each file (except standard input and output) is
opened with the name retrieved by the \caps{EXNAME} routine of the
\caps{SUPES} library \cite{supes}.

\section{Special Software}

\numbers\ is written in ANSI \caps{FORTRAN}-77.
\numbers\ uses the following software package:
\setlength{\itemsep}{\medskipamount} \begin{itemize}
\item the \caps{SUPES} package \cite{supes} which includes dynamic
memory allocation, a free-field reader, and \caps{\caps{FORTRAN}}
extensions.
\end{itemize}

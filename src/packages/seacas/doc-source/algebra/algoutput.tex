\chapter{The Output \exo{} Database} \label{chap:outexo}

The \exo{} database documentation can be accessed at
\url{http://gsjaardema.github.io/seacas/exodusII-new.pdf}. It contains
a full detailed description of \exo{}. This section presents an
overview of the concepts and structure of an \exo{} database. The
first part of the \exo{} database consists of the mesh
description which includes the nodal coordinates, the
element block information (including the element connectivity), the
node sets, and the side sets. The second part of the database contains
the time step information, including all the variable values for each
time step.

If nodes and/or elements have been deleted from the database with a
\cmd{ZOOM}, \cmd{FILTER}, or \cmd{VISIBLE} command, the entire output
database reflects the deletions and any node or element renumbering
caused by the deletions.

The output database mesh description is copied (with changes for
deletions) from the input database. The database title may be changed
with the \cmd{TITLE} command.

All QA records from the input database are copied to the output
database, and a record is added describing the current \caps{\PROGRAM}
run. All input database informational records are copied to the output
database.

All names on the output database are in uppercase and have all embedded
blanks removed. The coordinate and element block names from the input
database are converted and copied (with changes for deletions) to the
output database. The output variable names are assigned in the
equations.

The output database element variable truth table has an entry for each
output element variable which indicates whether the variable is defined
for each element block. This is determined by the input element variable
truth table, the equations, and the \cmd{BLOCKS} command.

The output time steps include the time step times and the output
variables for each time step. Each selected input time step is
processed; non-selected time steps are ignored. For
each time step, all variables are evaluated and written to the
output database.

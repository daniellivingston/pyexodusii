\chapter{Effect of Friction on Slapdown Severity}

\section{Introduction}

     Friction between the initial impact point and the target can have 
a significant effect on the secondary impact severity in a 
shallow angle slapdown event.  The coefficient of friction 
is a significant parameter influencing the importance of friction. 
However, the radius at which the friction force acts (distance between 
the contact point and the axial centerline of the object) also plays a 
significant role.  These two parameters are investigated here.

     The solid cylinder with length of 120 and radius to the edge of
the spring of 40 was used.  The mass was 80, the moment of inertia was
114,000, the initial vertical velocity was -527.5 and the initial
angle was 15$^\circ$.  The moderate stiffness linear elastic and the
nonlinear plastic springs described in the section on nose spring
effects (Chapter 5) were used.  

\begin{figure}
\vspace{3.5 in}
\caption{Effect of Friction on Slapdown Severity}
\end{figure}

\section{Coefficient of Friction}

     The results of a study on the effect of coefficient of friction
are shown in Table 6.1 and Figure 6.1.  
Friction had a much
greater effect for the linear elastic spring than for the nonlinear
plastic spring.  Two parameters contributed to this difference.  The
normal forces, and thus the frictional forces, were much higher for the
linear elastic spring. In addition, no energy was absorbed for the
linear elastic spring compared to the large amount absorbed in the
nonlinear plastic spring. Thus, energy dissipation due to friction
had a proportionally greater effect on the linear
elastic spring than on the nonlinear plastic spring. The increase in
tail velocity for coefficient of friction of 0.4 over that of 0.3 for
the linear elastic spring is due to the development of sufficient
frictional force that the nose sticks (nose velocity in the x
direction goes to zero).  When nose sticking occurs, no further energy
is dissipated by friction.  Therefore, more energy remains to
accelerate the tail resulting in higher tail velocities. 

\begin{figure}
\vspace{3.5 in}
\caption{Effect of Spring Radius (Moment Arm) on the Modification of
Slapdown Severity by Friction}
\end{figure}

\begin{table}
\begin{center}
\caption{Effect of Increasing Coefficient Friction on Secondary 
Impact}
\begin{tabular}{||c|c|c||}
\hline
\multicolumn{1}{||l|}{Linear Elastic Spring} & &\\
     Coefficient            &Tail                  &Tail\\
     of Friction          &Velocity            &Displacement\\
         0.0               &-979                   &6.028\\
         0.1               &-938                   &5.777\\
         0.2               &-891                   &5.490\\
         0.3               &-852                   &5.251\\
         0.4               &-865                   &5.331\\
\hline
\multicolumn{1}{||l|}{Nonlinear Plastic Spring} & & \\
     Coefficient            &Tail                  &Tail\\
     of Friction          &Velocity            &Displacement\\
         0.0               &-752                   &5.103\\
         0.1               &-733                   &4.993\\
         0.2               &-711                   &4.894\\
         0.3               &-690                   &4.756\\
         0.4               &-686                   &4.760\\
\hline
\end{tabular}
\end{center}
\end{table}

\section{Radius of Spring}

     The effect of spring radius (distance between the contact point
and the axial centerline of the object) 
is shown in Table 6.2 and Figure 6.2.
The secondary impact severity decreases almost linearly with
increasing spring radius for both linear and nonlinear springs.  As in
the coefficient of friction study (Section 6.2),
and for the same reasons, the
effect of spring radius was more pronounced for the linear elastic
spring than for the nonlinear plastic spring. The spring radius is
proportional to the moment arm over which the frictional forces act. 
Thus, increasing spring radius serves to retard rotation.  Retardation
of rotation decreases the secondary impact severity at the
expense of increasing the initial impact severity (energy is
shifted from secondary impact to primary impact). 

\begin{table}
\begin{center}
\caption{Effect of Increasing Spring Radius on Secondary Impact}
\begin{tabular}{||c|c|c||}
\hline
\multicolumn{1}{||l|}{Linear Elastic Spring} & &\\
\multicolumn{1}{||l|}{Coefficient of Friction 0.2} & &\\
       Radius               &Tail                  &Tail\\
      of Spring           &Velocity            &Displacement\\
         40.                &-891                  &5.490\\
         50.                &-871                  &5.365\\
         60.                &-849                  &5.232\\
         80.                &-802                  &4.942\\
        100.                &-748                  &4.614\\
        120.                &-689                  &4.250\\
        240.                &-455                  &2.828\\
        360.                &-261                  &1.661\\
\hline
\multicolumn{1}{||l|}{Nonlinear Plastic Spring} & & \\
\multicolumn{1}{||l|}{Coefficient of Friction 0.2} & & \\
       Radius               &Tail                  &Tail\\
      of Spring           &Velocity            &Displacement\\
         40.                &-711                  &4.894\\
         50.                &-702                  &4.832\\
         60.                &-692                  &4.787\\
         80.                &-670                  &4.652\\
        100.                &-645                  &4.504\\
        120.                &-617                  &4.325\\
        240.                &-502                  &3.630\\
        360.                &-430                  &3.069\\
\hline
\end{tabular}
\end{center}
\end{table} 

\section{Conclusions}

     It is difficult to make broad generalizations covering the
effects of friction on the shallow angle slapdown problem.  While
friction always serves to decrease the severity of the secondary
impact, in certain extreme cases, friction can increase the severity
of the initial impact sufficiently that the initial impact is more
severe than the secondary
impact.  In general, for geometries, impact
limiter behavior, and coefficients of friction anticipated in
transportation of radioactive materials, it will be conservative to
neglect friction in the analysis of shallow angle slapdown events.
However, if the impact limiter radius is large compared to the package
length or the coefficient of friction is anticipated to exceed 0.3,
neglecting friction can lead to significantly
unconservative initial impact predictions.

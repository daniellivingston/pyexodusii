\chapter{Scaling Behavior}

\section{Introduction}

     In the field of radioactive materials transportation cask design, 
testing of scale models is very important.  In order to successfully 
utilize the results of scale model tests, the appropriate parameters 
and techniques for extrapolating the results to full-scale behavior 
must be identified.  Because the accelerations of practical interest
are sufficiently high, the effects of a one G (gravity) field are 
ignored in the following discussion.

\section{Simplified Analysis for Linear Systems}

     For linear behavior, a simplified engineering approach to scaling
relations can be applied.  For radioactive materials transportation
cask applications, the linear accelerations at various locations along
the cask length tend to be the most important parameters.  These
accelerations are used primarily to assess inertial loads on bolted
closures, cask contents and other components.  Therefore, the
discussion below centers on the scaling relations for acceleration
values.  A more complete description of scaling relations can be found
in Table 7.1. 

Linear acceleration of the center-of-gravity is described
by the equation relating force ($F$), mass ($M$), 
and acceleration ($a$): 

\begin{equation}
                 F = M \cdot a.
\end{equation}

Angular acceleration about the center-of-gravity is similarly 
described by the equation relating torque ($T$), moment of 
inertia ($I$), and angular acceleration ($\alpha$): 

\begin{equation}
                 T = I \cdot \alpha.
\end{equation}

     For linear acceleration, the forces are proportional to the
square of the linear dimensions of the body.  The mass is proportional
to the cube of the linear dimensions.  Therefore, the acceleration of
a full-scale object should be half that of a half-scale object. 

     For angular acceleration, the torques are proportional to the
cube of the body scale 
(T = force $\cdot$ length, where force $\propto$ length$^{2}$).
The moment of inertia is
proportional to the fifth power of the body scale 
(I = mass $\cdot$ length$^{2}$, where mass $\propto$ length$^{3}$).
Therefore, the angular acceleration of a full-scale object
should be one fourth that of a half-scale object.  However, the linear
acceleration of the end of an object is equal to the angular
acceleration times a length dimension.  Thus, the linear acceleration
of the tip of a rotating full-scale object should again be half that
of a half-scale object.  Note, for simplicity of the argument and
because we are primarily concerned with the tangential component of
acceleration for shallow angle slapdown events, the inward component
of acceleration (centripetal acceleration) due to the square of the
angular velocity times the length has been neglected in this argument.
Inclusion of centripetal acceleration does not change the stated
conclusion. 

\section{Numerical Confirmation Using \SLAP}

     Nonlinear behavior of the structure at the impact locations is
much more common than linear behavior.  Nonlinear behavior arises from
both geometric and material properties sources.  The nonlinear
behavior is easily incorporated into numeric solution procedures such
as those found in the slapdown analysis program, \SLAP , discussed in
this report.  \SLAP\ will therefore be used to address the effects
of nonlinear behavior on the scaling relations for slapdown impact
events.  Two sets of analyses will be performed.  First, an analysis
of linear behavior will be compared to the results of the simple
engineering analysis described above.  Then, an analysis of nonlinear
behavior will determine the effects of nonlinearity on the scaling
relations derived for linear behavior. 

\subsection{Linear Springs}

     A simple solid cylinder was selected to demonstrate the
scaling relations.  No system of units was assumed, thus any
consistent units may be applied. In full-scale, the cylinder
was given
a mass of 80, a length of 120 units, and a radius of 
30 units resulting in a
moment of inertia of 114,000.
An identical spring at each end of the
cylinder had a travel of 10 units with a spring rate of 
600,000 and an identical
unloading modulus (that is the spring was linear on both loading and
unloading with no energy dissipation). This cylinder 
was impacted onto a
rigid target at an initial angle of 10$^\circ$ with an initial velocity
of 527.45.  This velocity was selected, 
based on the author's bias, because it is consistent, in units of
in/sec, with the NRC 30-foot free drop
hypothetical accident condition. A solid cylinder consistent 
with the half-
scale of the cylinder described above has a mass of 10,
a length of 60 units, 
and a radius of 15 units 
for a moment of inertia of 3563.  In half-scale,
the spring travel was 5 units with a spring rate of 
300,000 for both loading
and unloading.  The initial impact conditions were identical with 
those of the
full-scale cylinder.  The model description parameters are listed in
Table 7.1 for easy comparison. 

\begin{table}
\begin{center}
\caption{Comparison of Full- and Half-Scale Linear Spring Models}
\makeqnum
\begin{tabular}{||l|r|r|r||}
\hline
 &\multicolumn{1}{|c}{Full-Scale}
 &\multicolumn{1}{|c}{Half-Scale}
 &\multicolumn{1}{|c||}{Ratio}\\
 & & &\multicolumn{1}{|c||}{(Full/Half)}\\
Geometry and & & & \\
Initial Conditions: & & & \\
\quad Overall Length     &$???120.0$ &$??60.0$ &$?2.0?$\\
\quad Mass               &$????80.0$ &$??10.0$ &$?8.0?$\\
\quad Moment of Inertia  &$114000.0$ &$3563.0$ &$32.0?$\\
\quad Spring Constants: & & &\\
\quad \quad Loading             &$600000.$ &$300000.$ &$2.0?$\\
\quad \quad Unloading           &$600000.$ &$300000.$ &$2.0?$\\
\quad Initial Velocity  &$-527.5$ &$-527.5$ &$1.0?$\\
\quad Initial Angle     &$??10.0$ &$??10.0$ &$1.0?$\\
\hline
Results: & & & \\
Maximum Accelerations & & & \\
\quad    Nose            &$?85720.$ &$?171400.$ &$0.5?$\\
                    &$-67780.$ &$-135600.$ &$0.5?$\\
\quad    Tail            &$158400.$ &$316700.$ &$0.5?$\\
                    &$-36610.$ &$-73210.$ &$0.5?$\\
\quad    C.-G.            &$45300.$ &$90590.$ &$0.5?$\\
                     &$????0.$ &$????0.$ &$0.5?$\\
\quad    Angular          &$?1893.$ &$?7583.$ &$0.25$\\
                     &$-1023.$ &$-4090.$ &$0.25$\\
Maximum Velocities & & & \\
\quad    Nose              &$?536.4$ &$?536.4$ &$1.0?$\\
                      &$-527.5$ &$-527.5$ &$1.0?$\\
\quad    Tail          &$???1.7$ &$???1.7$ &$1.0?$\\
                  &$-983.2$ &$-983.2$ &$1.0?$\\
\quad    C.-G.            &$??56.7$ &$??56.7$  &$1.0?$\\
                     &$-527.5$ &$-527.5$  &$1.0?$\\
\quad    Angular            &$??0.0$ &$??0.0$ &$0.5?$\\
                       &$-12.7$ &$-25.3$ &$0.5?$\\
Maximum Displacements & & & \\
\quad    Nose            &$3.28$ &$1.64$ &$2.0?$\\
\quad    Tail            &$6.04$ &$3.02$ &$2.0?$\\
\hline
\end{tabular}
\end{center}
\end{table}

     \SLAP\ analysis results show that the sequence of
events occurs at twice the speed for the half-scale cylinder 
as for the
full-scale cylinder.  The displacements for the half-scale cylinder
are half the full-scale cylinder's
displacements.
Linear velocities are the same for
both half- and full-scale cylinders, and linear accelerations of the 
half-scale cylinder are double those of the full-scale cylinder.
The angular
velocities of the half-scale cylinder are double those of the full-scale
cylinder, and the angular accelerations for 
the half-scale cylinder are four times those
for the full-scale cylinder.  
Thus, if an object has a non-zero angular velocity
at impact, the half-scale object must have
double the initial angular velocity of the full-scale object, 
for proper scaling.  These
results are shown in Figures 7.1-7.3.  The maximum values of
displacement, velocity, and acceleration are shown in Table 7.1.

\begin{figure}
\vspace{3.5 in}
\caption{Nose, Center-of-Gravity, and Tail Displacements versus Time 
for Full- and Half-Scale Cylinders with Linear Springs}
\end{figure}

\begin{figure}
\vspace{3.5 in}
\caption{Nose, Center-of-Gravity, and Tail Velocities versus Time 
for Full- and Half-Scale Cylinders with Linear Springs}
\end{figure}

\begin{figure}
\vspace{3.5 in}
\caption{Nose, Center-of-Gravity, and Tail Accelerations versus Time 
for Full- and Half-Scale Cylinders with Linear Springs}
\end{figure}

\subsection{Nonlinear Springs}

     To investigate the nonlinear behavior, the cylinders described
above were again used.  The only differences were the springs.  For
the full-scale cylinder, 
a spring with an initial spring rate of 750,000 for
the initial travel of 2 units 
followed by a tangent spring rate of 0 for the
next 8 units 
was selected.  Thus the force curve rose linearly to a value of
1,500,000 
and then remained at that value for the remainder of the
compression.  An unloading modulus of 600,000 was again selected. 
Since unloading occurs linearly from the maximum load reached, the
spring selected has the potential to absorb a considerable amount of
energy.  The half-scale spring has an initial spring rate of 375,000
over the first unit 
of displacement. This force remains constant for the
next 4 units of displacement.  The unloading modulus is 300,000.  The
initial conditions for the impact event were identical with those
described above for the linear analysis.  The impact initial
conditions and model description are shown in Table 7.2. 

\begin{table}
\begin{center}
\caption{Comparison of Full- and Half-Scale Nonlinear Spring Models}
\makeqnum
\begin{tabular}{||l|r|r|r||}
\hline
 &\multicolumn{1}{|c}{Full-Scale}
 &\multicolumn{1}{|c}{Half-Scale}
 &\multicolumn{1}{|c||}{Ratio}\\
 & & &\multicolumn{1}{|c||}{(Full/Half)}\\
Geometry and & & & \\
Initial Conditions: & & & \\
\quad Overall Length     &$???120.0$ &$??60.0$ &$?2.0?$\\
\quad Mass               &$????80.0$ &$??10.0$ &$?8.0?$\\
\quad Moment of Inertia  &$114000.0$ &$3563.0$ &$32.0?$\\
\quad Spring Constants: & & &\\
\quad \quad Loading section 1   &$750000.?$ &$375000.?$ &$2.0?$\\
\quad \quad Loading section 2   &$     0.?$ &$     0.?$ &$2.0?$\\
\quad \quad Unloading           &$600000.?$ &$300000.?$ &$2.0?$\\
\quad Initial Velocity  &$-527.5$ &$-527.5$ &$1.0?$\\
\quad Initial Angle     &$??10.0$ &$??10.0$ &$1.0?$\\
\hline
Results: & & & \\
Maximum Accelerations & & & \\
\quad    Nose    &$65630.?$ &$131300.?$ &$0.5?$\\
            &$-28760.?$ &$-57530.?$ &$0.5?$\\
\quad    Tail    &$66260.?$ &$132500.?$ &$0.5?$\\
            &$-28130.?$ &$-56260.?$ &$0.5?$\\
\quad    C.-G.   &$18750.?$ &$37500.?$ &$0.5?$\\
            &$0.?$    &$0.?$    &$0.5?$\\
\quad    Angular &$788.5$  &$3154.$  &$0.25$\\
            &$-781.5$ &$-3126.$ &$0.25$\\
Maximum Velocities & & & \\
\quad    Nose &$409.5???$ &$409.5???$ &$1.0?$\\
         &$-527.5???$ &$-527.5???$ &$1.0?$\\
\quad    Tail &$7.5???$ &$7.5???$ &$1.0?$\\
         &$-928.8???$ &$-928.8???$ &$1.0?$\\
\quad   C.-G. &$6.6???$ &$6.6???$ &$1.0?$\\
         &$-527.5???$ &$-527.5???$ &$1.0?$\\
\quad Angular &$0.0158$ &$???0.0316$ &$0.5?$\\
         &$-11.15??$ &$-22.3???$ &$0.5?$\\
Maximum Displacements & & & \\
\quad    Nose &$3.16$ &$1.58$ &$2.0?$\\
\quad    Tail &$7.53$ &$3.77$ &$2.0?$\\
\hline
\end{tabular}
\end{center}
\end{table}

     The results for the nonlinear springs showed exactly the same 
behavior as described for the linear springs.  Nonlinear behavior is 
shown in Figures 7.4-7.6 and the maximum values in Table 7.2.

\begin{figure}
\vspace{3.5 in}
\caption{Nose, Center-of-Gravity, and Tail Displacements versus Time 
for Full- and Half-Scale Cylinders with Nonlinear Springs}
\end{figure}

\begin{figure}
\vspace{3.5 in}
\caption{Nose, Center-of-Gravity, and Tail Velocities versus Time 
for Full- and Half-Scale Cylinders with Nonlinear Springs}
\end{figure}

\begin{figure}
\vspace{3.5 in}
\caption{Nose, Center-of-Gravity, and Tail Accelerations versus Time 
for Full- and Half-Scale Cylinders with Nonlinear Springs}
\end{figure}

\section{Conclusions}

     In this section, we have shown that the results of impact testing
scale models can easily be related to the behavior of full-scale
objects.  Model displacements can be related to full-scale object
displacements by multiplying by the model scale.  Model linear
accelerations can be related to full-scale object linear accelerations
by dividing by the model scale.  Model angular accelerations can be
related to those of the full-scale object by dividing by the square of
the model scale. Linear velocities are identical between model and
full-scale objects regardless of scale.  Angular velocities scale in
the manner of linear accelerations.  The scaling relations for
velocities indicate that the initial conditions for a model impact
test should be identical to those expected for the full-scale event
except for initial angular velocity.  Initial angular velocity for the
model should be inversely proportional to the scale of the model used.
These scaling relations are summarized in Table 7.2.
\begin{table}
\begin{center}
\caption{Summary of Relationships for Scale Model Testing}
\makeqnum
\begin{tabular}{||l|c||}
\hline
\multicolumn{1}{|c}{Parameter}
 &\multicolumn{1}{|c||}{Scaling Relationships}\\
Geometry and & \\
Initial Conditions: &\\
\quad Overall Length     & $l_{sm} = l_{fs} \times (Scale)^{1}$\\
\quad Mass               & $M_{sm} = M_{fs} \times (Scale)^{3}$\\
\quad Moment of Inertia  & $I_{sm} = I_{fs} \times (Scale)^{5}$\\
\quad Spring Constants   & $K_{sm} = K_{fs} \times (Scale)^{1}$\\
\quad Initial Velocity   & $V_{sm} = V_{fs} \times (Scale)^{0}$\\
\quad Initial Angle      & $\theta _{sm} = \theta _{fs} \times 
(Scale)^{0}$\\
\hline
Results: & \\
\quad    Linear Accelerations  & $a_{sm} = a_{fs} \times (Scale)^{-1}$\\
\quad    Angular Accelerations & $\alpha _{sm} = \alpha _{fs} \times 
(Scale)^{-2}$\\
\quad    Linear Velocities     & $V_{sm} = V_{fs} \times (Scale)^{0}$\\
\quad    Angular Velocities    & $\omega _{sm} = \omega _{fs} \times 
(Scale)^{-1}$\\
\quad    Linear Displacements  & $\Delta _{sm} = \Delta _{fs} \times 
(Scale)^{1}$\\
\quad    Angular Displacements & $\theta _{sm} = \theta _{f} \times 
(Scale)^{0}$\\
\hline
\end{tabular}
\end{center}
\end{table}


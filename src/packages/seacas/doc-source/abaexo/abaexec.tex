\chapter{Executing \caps{\PROGRAM}} \label{chap:exec}

The details of executing \caps{\PROGRAM} are dependent on the system
being used. The system manager of any system that runs \caps{\PROGRAM}
should provide a supplement to this manual that explains how to run the
program on that particular system. Site supplements for all currently
supported systems are in Appendix~\ref{appx:site}.


\section{Execution Files} \label{exec:files}

The table below summarizes \caps{\PROGRAM}'s file usage.

\begin{center} \begin{tabular}{||l|c|c|c||}
\hline
\multicolumn{1}{||c}{Description} &
\multicolumn{1}{|c}{Unit Number} &
\multicolumn{1}{|c}{Type} &
\multicolumn{1}{|c||}{File Format} \\
\hline
User input & standard input & input & Section~\ref{chap:command} \\
User output & standard output & output & ASCII \\
ABAQUS database & 8 & input & Appendix~\ref{appx:abaqus} \\
EXODUS database & 11 & output & Appendix~\ref{appx:exodus} \\
Scratch file & 20 & scratch & binary \\
\hline
\end{tabular} \end{center}

All files must be connected to the appropriate unit before
\caps{\PROGRAM} is run. Each file (except standard input and output) is
opened with the name retrieved by the \caps{EXNAME} routine of the
\caps{SUPES} library \cite{bib:supes}.

\section{Special Software}

\caps{\PROGRAM} is written in ANSI \caps{FORTRAN}-77 \cite{bib:f77}
with the exception of the following system-dependent features:
\setlength{\itemsep}{\medskipamount} \begin{itemize}
\item the \caps{OPEN} options for the files.
\end{itemize}

\caps{\PROGRAM} uses the following software packages:
\setlength{\itemsep}{\medskipamount} \begin{itemize}
\item the \caps{ABAQUS} library package which opens and reads the
\caps{ABAQUS} database and
\item the \caps{SUPES} package \cite{bib:supes} which includes dynamic
memory allocation, a free-field reader, and \caps{FORTRAN} extensions.
\end{itemize}
